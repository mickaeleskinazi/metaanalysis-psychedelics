\section{Discussion}

Our meta-analysis integrates 508 randomized contrasts spanning 149 LSD, 197 MDMA, and 162 psilocybin session-level comparisons, providing a granular view of adverse event (AE) risk during and after psychedelic-assisted therapy (Table~\ref{tab:dr-global-by-molecule}).
Across molecules, the global dose--response models confirm that higher session doses increase AE odds, with psilocybin showing the steepest gradient ($Q_M=19.13$, $p=1.22\times10^{-5}$), LSD a robust but more moderate slope ($Q_M=11.36$, $p=7.49\times10^{-4}$), and MDMA a weaker yet significant trend ($Q_M=5.16$, $p=2.31\times10^{-2}$).
Follow-up models showed no persistent positive slopes for either LSD or MDMA (Figure~\ref{fig:dr-session-followup}), indicating that acute dose escalation does not translate into sustained AE burden.

\subsection{Translating dose--response findings into clinical guardrails}

The AE-specific regressions (Table~\ref{tab:dr-ae-by-molecule-session}) demonstrate that only a limited subset of adverse experiences scales directly with dose, and those events are predominantly somatic.
For LSD, nausea, headache, and visual illusions intensified with higher microgram exposures.
These results echo the structured monitoring protocols proposed by Hinkle et~al. (2024) and reinforce that once psychedelic threshold dosing is reached, mental-state perturbations behave more like on/off phenomena than gradated toxicities.

MDMA, in contrast, manifested dose-linked dizziness and headache.
Both slopes rose linearly across the 50--125\,mg range, consistent with the sympathetic activation highlighted by Colcott et~al. (2024).
Clinically, this supports pre-session hydration, electrolyte monitoring, and staged titration for participants with migraine or vestibular vulnerability.

Psilocybin displayed the most pronounced dose response at the AE level: a non-linear surge in fatigue above 25\,mg and a linear increase in hypertension risk ($p=4.07\times10^{-3}$).
These findings dovetail with the cardiovascular screening recommendations articulated by Breeksema et~al. (2022).
No analyzed molecule exhibited significant negative slopes, and ayahuasca trials lacked sufficient dosing variability for inference, highlighting a continued evidence gap for traditional brew formulations.

\subsection{Temporal dynamics of adverse events}

Forest analyses complement the dose--response slopes by revealing when AE risks are clinically salient.
Only five molecule-specific signals remained significant after multiplicity correction, and all were confined to the dosing day (Table~\ref{tab:ae-transition}).
For LSD, session elevations in nausea and headache fully resolved by the first follow-up assessment ($p>0.60$).
MDMA produced session-only increases in headache and depressed mood, mirroring the controlled follow-up data summarized by Colcott et~al. (2024).
For psilocybin, session fatigue was the sole significant pooled signal, aligning with patient reports of transient lethargy.

Dose--response contrasts between session and follow-up windows add nuance to these categorical findings.
LSD retained a significant positive slope during the session but exhibited a modest negative slope at follow-up, suggesting that higher doses may expedite resolution of mild aftereffects.
MDMA’s follow-up slope was null, reinforcing that the drug’s sympathomimetic stress is tightly temporally coupled to the dosing period.
These observations emphasize that intensive medical monitoring should be concentrated on the dosing day, with targeted rather than routine pharmacologic prophylaxis afterward.

\subsection{Matching molecules to patient archetypes}

To translate these quantitative patterns into actionable clinical decision-making, we propose the framework in Table~\ref{tab:clinical-matching}.
The table integrates our AE signals with common patient archetypes drawn from contemporary psychedelic-assisted therapy programs: older adults with cardiovascular disease, individuals with high anxiety or trauma-related arousal, and patients managing complex polypharmacy.
For each profile we highlight the molecule whose AE profile is most congruent and recommend mitigating strategies when risk signals suggest caution.

\begin{table}[htbp]
  \centering
  \caption{Results-informed molecule selection and monitoring recommendations for representative clinical profiles.}
  \label{tab:clinical-matching}
  \resizebox{\textwidth}{!}{%%
  \begin{tabular}{p{0.23\textwidth}p{0.24\textwidth}p{0.28\textwidth}p{0.21\textwidth}}
    \toprule
    Patient profile & Key safety concerns & Preferred molecule / dose strategy & Monitoring and mitigation \\
    \midrule
    Older adult with controlled hypertension & Psilocybin-driven dose--dependent hypertension; MDMA sympathomimetic load & LSD at the lower end of therapeutic dosing (50--100\,\textmu g) to minimize cardiovascular swings while avoiding psilocybin-induced pressor responses & Baseline ECG; intra-session blood-pressure cycling every 20\,min; ondansetron availability for nausea; reinforce hydration post-session \\
    Middle-aged patient with cardiovascular disease and polypharmacy & Drug--drug interactions via CYP2D6 (MDMA) or serotonergic load (psilocybin) & Carefully titrated psilocybin (15--20\,mg) when antihypertensive regimen is stable; avoid MDMA in beta-blocker or SSRI users & Continuous BP and pulse oximetry during session; collaborate with prescribing cardiologist; schedule day-2 check-in for delayed fatigue monitoring \\
    Anxiety-prone trauma survivor & Risk of MDMA-associated acute dysphoria; LSD visual intensity & MDMA with conservative titration (80\,mg plus optional 40\,mg booster) paired with preparatory coping skills; LSD reserved for patients comfortable with perceptual shifts & Real-time affect monitoring; rapid access to grounding interventions; plan follow-up psychotherapy within 24 hours to address transient depressed mood \\
    Patient with chronic pain and sleep fragility & Concern about psilocybin fatigue and LSD headaches & Psilocybin at 20--25\,mg with scheduled rest day and emphasis on sleep hygiene; avoid MDMA due to headache prominence & Provide post-session fatigue diary; employ non-pharmacologic analgesia; evaluate sleep quality at 48 hours and intervene if insomnia emerges \\
    Geriatric patient with mild cognitive impairment & Tolerance of intense perceptual content; polypharmacy with anticholinergics & Low-dose psilocybin (10--15\,mg) when cardiovascular profile allows; LSD only in highly structured settings with visual reframing support & Cognitive orientation checks every 30\,min; caregiver presence during integration; reinforce hydration and electrolyte balance \\
    \bottomrule
  \end{tabular}}
\end{table}

The recommendations highlight three guiding principles: (i) cardiovascular comorbidity warrants caution with psilocybin because of its clear hypertension and fatigue signals, (ii) MDMA’s session-only headache and mood effects favor dose titration and psychological containment over extended medical surveillance, and (iii) LSD’s GI-dominant risk profile argues for proactive antiemetic strategies instead of excluding anxious or somatically sensitive patients.

\subsection{Implications for protocol design}

The convergence of dose--response and forest analyses supports several refinements to clinical protocols.
Dose selection should prioritize the lowest effective range that meets therapeutic objectives while acknowledging the steepness of psilocybin’s fatigue curve beyond 25\,mg.
Staged escalation---especially for MDMA and LSD---can individualize tolerability without materially compromising efficacy, and the absence of persistent AE signals argues for targeted, patient-specific follow-up rather than uniform medicalization of the integration phase.

Adverse-event taxonomies should continue to disaggregate somatic and psychological domains.
Somatic AEs (nausea, headache, hypertension) are far more likely to exhibit dose gradients than psychological phenomena, informing both informed-consent language and mitigation toolkits such as vestibular stabilization exercises or prophylactic triptans for headache-prone patients (Hinkle et~al., 2024).

\subsection{Limitations and future research}

Several limitations temper these conclusions.
Follow-up data remain sparse, especially for psilocybin and ayahuasca, limiting detection of delayed AEs.
Study heterogeneity---from psychotherapy frameworks to AE ascertainment windows---likely attenuated power for certain endpoints, and most trials excluded participants with uncontrolled medical illness, meaning that recommendations for complex comorbidity rely on extrapolation from observational work (Breeksema et~al., 2022; Colcott et~al., 2024).
Future work should prioritize harmonized AE reporting standards, head-to-head comparisons of titration schedules, and prospective registries that capture long-term safety profiles across diverse patient populations.

\subsection{Conclusion}

In sum, this meta-analysis delineates a clear, molecule-specific safety landscape for psychedelic-assisted therapy.
Dose-dependent AE risk is concentrated in somatic domains during the dosing session, with minimal evidence of persistent post-session toxicity.
Clinicians can leverage these findings to tailor molecule selection, dosing, and monitoring to the needs of elderly patients, individuals with cardiovascular comorbidities, and those prone to anxiety or fatigue.
By integrating rigorous quantitative synthesis with pragmatic clinical guardrails, we move closer to a precision-matched deployment of psychedelics that maximizes therapeutic benefit while safeguarding patient well-being.
