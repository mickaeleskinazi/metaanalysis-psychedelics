\section{Discussion}

Our meta-analysis integrates 508 randomized contrasts spanning 149 LSD, 197 MDMA, and 162 psilocybin session-level comparisons,
providing a granular view of adverse event (AE) risk during and after psychedelic-assisted therapy (Table~\ref{tab:dr-global-by-molecule}).
Across molecules, the global dose--response models confirm that higher session doses increase AE odds, with psilocybin showing the steepest gradient ($Q_M=19.13$, $p=1.22\times10^{-5}$), LSD a robust but more moderate slope ($Q_M=11.36$, $p=7.49\times10^{-4}$), and MDMA a weaker yet significant trend ($Q_M=5.16$, $p=2.31\times10^{-2}$).
Follow-up models showed no persistent positive slopes for either LSD or MDMA (Figure~\ref{fig:dr-session-followup}), indicating that acute dose escalation does not translate into sustained global AE burden, even though categorical signals for specific AEs can linger.

\subsection{Dose-responsive safety signals should drive individualized dosing}

The AE-specific regressions (Table~\ref{tab:dr-ae-by-molecule-session}) demonstrate that only a limited subset of adverse experiences scales directly with dose, and those events are predominantly somatic.
For LSD, nausea, headache, and visual illusions intensified with higher microgram exposures, warranting proactive hydration, non-opioid analgesia, and gastrointestinal support when titrating above 100\,\textmu g.
MDMA manifested dose-linked dizziness and headache across the 50--125\,mg range; conservative starting doses with optional boosters allow clinicians to gauge sympathetic activation in real time before escalating.
Psilocybin displayed the most pronounced dose response: a non-linear surge in fatigue above 25\,mg and a linear increase in hypertension risk ($p=4.07\times10^{-3}$).
These gradients support tailoring psilocybin dose to the intended therapeutic depth and to the patient’s cardiovascular reserve, reserving higher (25--30\,mg) exposures for individuals who can tolerate transient pressor responses.

Dose-dependent somatic burden should be integrated into preparation, consent, and integration.
Because bodily discomfort scales more predictably with dose than psychological content, preparatory visits should rehearse grounding strategies---mindfulness-based cognitive therapy, acceptance-based skills, and compassion-focused practices---that help patients metabolize nausea, dizziness, or blood-pressure lability as anticipated and manageable phenomena.
In-session teams can reinforce agency by linking emergent sensations to the dose plan, while post-session integration consolidates these experiences into the patient’s narrative so that transient AEs become part of the therapeutic learning rather than sources of distress.

\subsection{Timing matters: pairing acute vigilance with structured follow-up}

Forest analyses complement the dose--response slopes by revealing when AE risks are clinically salient.
Molecule-specific forest plots showed that a handful of categorical AE signals persist beyond the dosing day despite flat follow-up dose gradients (Figure~\ref{fig:forest-combined}).
For LSD, follow-up odds ratios remained elevated for \textit{headache} (OR\,=\,10.70, $p<10^{-4}$), \textit{fatigue} (OR\,=\,14.21, $p<10^{-3}$), and \textit{attention disturbance} (OR\,=\,7.95, $p=2.0\times10^{-3}$).
MDMA displayed sustained elevations for \textit{anxiety} (OR\,=\,4.93, $p=2.6\times10^{-3}$), \textit{sleep disorder} (OR\,=\,3.83, $p=3.8\times10^{-4}$), and \textit{fatigue} (OR\,=\,11.85, $p<10^{-3}$).
Psilocybin’s significant signals remained session-bound, with fatigue resolving by the earliest follow-up visits.

These dynamics call for a two-tiered safety plan.
First, intensive medical monitoring---blood-pressure cycling, temperature and hydration checks, and real-time affect tracking---should be concentrated on the dosing day, where dose-dependent somatic AEs cluster.
Second, integration visits within 24--72 hours should explicitly screen for the molecule-specific follow-up signals identified in the forest models (e.g., sleep disruption after MDMA, cognitive slowing after LSD) and should provide targeted behavioral or pharmacologic supports when needed.
Transparent counseling about this timeline helps patients anticipate and normalize lingering symptoms, strengthening adherence to recovery plans and reducing unsupervised medication use.

The psilocybin suicidality signal warrants particular attention.
Both Goodwin et~al. (2022) and Bogenschutz et~al. (2022) enrolled participants with severe, recurrent depression or alcohol use disorder and high baseline suicidality, contributing to an elevated session OR for suicidal ideation (OR\,=\,4.30, $p=0.014$).
Although the dose--response models did not identify a gradient for suicidality, these trials illustrate the need for intensified post-session observation, crisis-response rehearsal, and coordination with existing mental-health providers whenever psilocybin is administered to patients with active suicidal risk.

\subsection{Extending guardrails to patients who fall outside RCT inclusion criteria}

The controlled trials synthesized here excluded many individuals who populate routine clinical practice.
Older adults, patients with multimorbidity, and those on complex polypharmacy regimens were rarely enrolled: only two trials encompassed ayahuasca, none reported follow-up data for psilocybin, and fewer than one in five contrasts involved participants over age 60 (Table~\ref{tab:study_summary_wide}).
Such sampling produces a safety profile that is internally valid yet not fully generalizable.
Psilocybin’s hypertension signal and MDMA’s sympathomimetic load raise unresolved questions for patients with structural heart disease or autonomic neuropathy, while LSD’s delayed fatigue could be functionally disabling for caregivers or individuals with limited recovery time.

Bridging this gap requires aligning dosing and monitoring with real-world comorbidity.
Pre-session work should incorporate detailed cardiometabolic and psychiatric risk stratification, medication reconciliation to identify cytochrome or serotonergic interactions, and collaborative planning with primary or specialty providers.
Where evidence is thin, observational registries and pragmatic trials that admit higher-risk cohorts will be critical to refine the guardrails proposed here and to document how adaptive dosing performs outside the narrow eligibility of traditional randomized controlled trials.

\subsection{Matching molecules to patient archetypes}

To translate these quantitative patterns into actionable clinical decision-making, we propose the framework in Table~\ref{tab:clinical-matching}.
The table integrates our AE signals with common patient archetypes drawn from contemporary psychedelic-assisted therapy programs: older adults with cardiovascular disease, individuals with high anxiety or trauma-related arousal, and patients managing complex polypharmacy.
For each profile we highlight the molecule whose AE profile is most congruent and recommend mitigating strategies when risk signals suggest caution.

\begin{table}[htbp]
  \centering
  \caption{Results-informed molecule selection and monitoring recommendations for representative clinical profiles.}
  \label{tab:clinical-matching}
  \resizebox{\textwidth}{!}{%%
  \begin{tabular}{p{0.23\textwidth}p{0.24\textwidth}p{0.28\textwidth}p{0.21\textwidth}}
    \toprule
    Patient profile & Key safety concerns & Preferred molecule / dose strategy & Monitoring and mitigation \\
    \midrule
    Older adult with controlled hypertension & Psilocybin-driven hypertension; MDMA sympathomimetic load; LSD follow-up headache/fatigue & LSD at 50--75\,\textmu g to blunt pressor swings while accepting manageable delayed somatic drag & Baseline ECG; intra-session blood-pressure cycling every 20\,min; schedule day-2 analgesia/sleep check-in; ondansetron availability for nausea \\
    Middle-aged patient with cardiovascular disease and polypharmacy & Drug--drug interactions via CYP2D6 (MDMA) or serotonergic load (psilocybin); external reports of MDMA arrhythmia & Carefully titrated psilocybin (15--20\,mg) only with cardiology clearance; avoid MDMA in beta-blocker or SSRI users; consider LSD microdosing protocols if fatigue intolerable & Continuous BP and pulse oximetry during session; coordinate with prescribing cardiologist; 48-hour follow-up for blood-pressure and fatigue surveillance \\
    Anxiety-prone trauma survivor & Post-session MDMA anxiety/sleep disruption; LSD visual intensity & MDMA with conservative titration (80\,mg plus optional 40\,mg booster) plus coping plans; arrange sleep hygiene coaching and as-needed non-benzodiazepine anxiolytics & Real-time affect monitoring; overnight access to support line; structured follow-up psychotherapy within 24 and 72 hours to manage lingering anxiety and insomnia \\
    Patient with chronic pain and sleep fragility & Concern about psilocybin fatigue and suicidal ideation; MDMA follow-up fatigue; LSD delayed headache despite acute sleep relief & Psilocybin at 20\,mg if suicidality low and caregiver available; alternatively, low-dose LSD with preplanned non-opioid analgesia & Provide post-session fatigue and mood diary; arrange sleep quality check at 48 hours; ensure crisis plan for depressive symptoms when using psilocybin \\
    Geriatric patient with mild cognitive impairment & Polypharmacy with anticholinergics; susceptibility to LSD attention disturbance & Low-dose psilocybin (10--15\,mg) with cardiovascular screening; reserve LSD for environments offering cognitive pacing and caregiver reinforcement & Cognitive orientation checks every 30\,min; caregiver presence during integration; reinforce hydration, electrolyte balance, and scheduled rest breaks \\
    \bottomrule
  \end{tabular}}}
\end{table}

The recommendations highlight three guiding principles: (i) cardiovascular comorbidity warrants caution with psilocybin because of its clear hypertension and fatigue signals as well as the concentration of suicidality monitoring in treatment-resistant depression trials, (ii) MDMA’s lingering anxiety, sleep disruption, and fatigue necessitate structured multi-day follow-up despite the absence of a follow-up dose gradient, and (iii) LSD’s combination of gastrointestinal burden and delayed headache/fatigue argues for proactive antiemetic and analgesic strategies while capitalizing on its acute sleep-stabilizing effects for patients with insomnia risk.

\subsection{Anticipating adverse events beyond controlled settings}

Although serious complications were absent from the randomized contrasts, case reports and observational data from less supervised contexts describe psychotic decompensation, hallucinogen persisting perception disorder, and other enduring neuropsychiatric sequelae, occasionally accompanied by autonomic instability or hypertensive crises \cite{Oosterhof2005_HPPDcase,Smith2023_PsilocybinPsychosis,Halpern2022_HPPDcases,turn0search9_case}.
These rare but consequential outcomes highlight why informed consent must emphasize both intra-session and delayed risks, including the possibility of emergent psychiatric symptoms that may require urgent evaluation.
Clinical protocols should therefore extend safety nets beyond the research milieu: structured crisis-response plans, caregiver education about delayed warning signs, and ready access to specialist consultation can mitigate harm if severe AEs arise despite protocol adherence.
As psychedelic services diffuse into semi-regulated or underground markets, continuous pharmacovigilance, standardized AE reporting, and prospective registries will be essential to detect infrequent events that fall below the resolution of randomized trials.

\subsection{Implications for protocol design}

The convergence of dose--response and forest analyses supports several refinements to clinical protocols.
Dose selection should prioritize the lowest effective range that meets therapeutic objectives while acknowledging the steepness of psilocybin’s fatigue curve beyond 25\,mg.
Staged escalation---especially for MDMA and LSD---can individualize tolerability without materially compromising efficacy, yet the categorical persistence of anxiety and sleep disruption after MDMA underscores the importance of at least 48~hours of structured follow-up.
Likewise, LSD programs should incorporate delayed check-ins that assess headache, fatigue, and attentional drift, even when acute monitoring is uneventful.

Adverse-event taxonomies should continue to disaggregate somatic and psychological domains.
Somatic AEs (nausea, headache, hypertension) are far more likely to exhibit dose gradients than psychological phenomena, informing both informed-consent language and mitigation toolkits such as vestibular stabilization exercises or prophylactic triptans for headache-prone patients (Hinkle et~al., 2024).
Psychological endpoints, by contrast, often manifest categorically: the suicidality signals seen in psilocybin depression trials likely reflect population selection and intensive symptom probing rather than pharmacologic induction, yet they compel explicit crisis planning during integration visits and demonstrate why depression programs require rapid-response safety nets.

\subsection{Limitations and future research}

Several limitations temper these conclusions.
Follow-up data remain sparse, especially for psilocybin and ayahuasca, limiting detection of delayed AEs.
Study heterogeneity---from psychotherapy frameworks to AE ascertainment windows---likely attenuated power for certain endpoints, and most trials excluded participants with uncontrolled medical illness, meaning that recommendations for complex comorbidity rely on extrapolation from observational work (Breeksema et~al., 2022; Colcott et~al., 2024).
Future work should prioritize harmonized AE reporting standards, head-to-head comparisons of titration schedules, prospective registries that capture long-term safety profiles across diverse patient populations, and qualitative follow-up that distinguishes pharmacologic persistence from psychological processing effects.

\subsection{Conclusion}

In sum, this meta-analysis delineates a clear, molecule-specific safety landscape for psychedelic-assisted therapy.
Dose-dependent AE risk is concentrated in somatic domains during the dosing session, yet forest-derived follow-up signals highlight persistent anxiety, sleep disruption, headache, and fatigue for select molecules.
Clinicians can leverage these findings to tailor molecule selection, dosing, and monitoring to the needs of elderly patients, individuals with cardiovascular comorbidities, and those prone to anxiety or fatigue, pairing acute vigilance with targeted follow-up.
By integrating rigorous quantitative synthesis with pragmatic clinical guardrails, we move closer to a precision-matched deployment of psychedelics that maximizes therapeutic benefit while safeguarding patient well-being across both session and recovery phases.
