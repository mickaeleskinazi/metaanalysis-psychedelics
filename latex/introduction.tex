Psychedelic-assisted therapy has re-emerged as a promising intervention for psychiatric disorders such as depression, anxiety, and post-traumatic stress disorder (PTSD), with lysergic acid diethylamide (LSD), psilocybin, 3,4-methylenedioxymethamphetamine (MDMA), and ayahuasca at the forefront of clinical research \cite{breeksema2022adverse,hinkle2024adverse}. Recent trials demonstrate encouraging efficacy across diagnostic categories, but also report diverse adverse events (AEs) that must be rigorously characterized to ensure safety in clinical practice \cite{colcott2024mdma,hinkle2024adverse}. This meta-analysis synthesizes only \emph{controlled human studies}; recreational or uncontrolled uses are outside its scope.

Understanding AEs in the clinical context is essential as psychedelic-assisted therapies move from research laboratories to hospitals and outpatient clinics. Patients seeking these interventions often present with overlapping indications treatment resistant depression, generalized anxiety, PTSD and frequently with medical or psychiatric comorbidities that heighten their vulnerability to even transient physiological or psychological disturbances \cite{scala2024revival,sarparast2022interactions}. Detailed AE mapping helps refine patient selection, optimize dosing, and establish robust safety protocols in settings where frailty, cardiovascular risk, or polypharmacy may alter drug tolerability \cite{breeksema2022adverse,hinkle2024adverse}. For example, MDMA’s sympathomimetic properties can pose cardiovascular risks in older adults or hypertensive patients, whereas psilocybin and LSD may provoke acute anxiety in highly anxious or trauma-prone individuals \cite{colcott2024mdma,holze2022direct,sarparast2022interactions}.

Across modern controlled trials, serotonergic psychedelics are generally well tolerated under supervision. Psilocybin and LSD typically elicit mild, transient somatic AEs such as nausea, dizziness, and headache, alongside short-lived psychological distress \cite{hinkle2024adverse,scala2024revival,breeksema2022adverse}. A head-to-head comparison found broadly overlapping subjective and physiological effects between psilocybin and LSD, though LSD’s longer duration may increase logistical and cardiovascular demands \cite{holze2022direct}. MDMA-assisted psychotherapy, by contrast, is characterized by dose-dependent autonomic and physical AEs—jaw clenching, perspiration, and elevated blood pressure yet serious or life-threatening events remain rare under controlled dosing and monitoring \cite{colcott2024mdma,sarparast2022interactions}. Ayahuasca, less studied in randomized settings, shows consistent gastrointestinal and cardiovascular AEs.\cite{breeksema2022adverse,feulner2023anxiety}.

Prior reviews have established overall tolerability but leave important gaps. Most aggregate AEs qualitatively or without modeling dose effects, and few distinguish between acute (session) and delayed (follow-up) events, even though these windows likely reflect distinct physiological and psychological mechanisms \cite{hinkle2024adverse,breeksema2022adverse}. Dose–response relationships are also inconsistently evaluated, despite evidence that some AEs increase monotonically with dose, while others show threshold patterns \cite{holze2022direct,scala2024revival}. Addressing these methodological gaps, combining dose–response modeling with time-window stratification, is therefore critical to refine pharmacovigilance and clinical guidance.

Recent meta-analyses confirm a generally favorable safety profile for classic psychedelics and MDMA under clinical supervision but emphasize molecule-specific AE signatures \cite{hinkle2024adverse,colcott2024mdma}. Such differences may guide molecule selection and contraindications: psilocybin’s shorter duration and milder cardiovascular load may make it preferable for medically fragile patients, while LSD or MDMA could require stricter exclusion criteria in those with anxiety sensitivity or cardiac disease \cite{sarparast2022interactions,omidi2025lsd}. As psychedelic treatments move toward broader implementation, clinicians will need quantitative, molecule-specific evidence linking AEs to dose and time course to inform risk management and patient counseling.

Here, we perform a dose–response meta-analysis of AEs across molecules, stratified by session and follow-up windows, to provide a comparative safety framework for psychedelic-assisted therapies in real-world clinical settings.